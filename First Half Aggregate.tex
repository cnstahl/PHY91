\documentclass[12pt, letterpaper]{article}

\usepackage[utf8]{inputenc}
\usepackage[margin=1in]{geometry}

\usepackage{amsmath, mathtools}
\usepackage{amssymb}
\usepackage{enumerate}
\usepackage{etoolbox}

\usepackage{physics}
\usepackage{braket}
\usepackage{bbold}

\renewcommand{\thesubsection}{\alph{subsection})}
\preto{\section}{\setcounter{subsection}{0}}
\newcommand{\hilb}{\mathfrak{H}}
\newcommand{\minus}[1]{\bar{#1}}


\title{PHY 91: Massive, Aggregate Problem Set}
\author{Calvin Chen, Bendeguz Offertaler, Charles Stahl}
\date{November 12th, 2016}
\begin{document}
\maketitle
\section*{October 10th}
\begin{enumerate}
    \item[] \textbf{G\&Y: 3.8}
    
\begin{enumerate}[(a)]
    \item 


Since $m_{max} = m_{1,max} + m_{2,max} + m_{3,max} = 3, \hspace{1mm} J_{max} = 3$.

\begin{align}
\hilb_1\otimes\hilb_1\otimes\hilb_1 &= \hilb_1 \otimes\left(\hilb_0\oplus
    \hilb_1 \oplus\hilb_2\right)\\
&= (\hilb_1\otimes\hilb_0)\oplus(\hilb_1\otimes\hilb_1)\oplus 
    (\hilb_1\otimes\hilb_2)\\
&= (\hilb_1)\oplus(\hilb_0\oplus\hilb_1\oplus_2) \oplus
    (\hilb_1\oplus\hilb_2\oplus\hilb_3)
\end{align}
Remembering that the degeneracy $deg(\hilb_J) = 2J+1$, we have
\begin{align}
&J = 0, \hspace{2mm}d_0 = 1*1 = 1\\
&J = 1, \hspace{2mm}d_1 = 3*3 = 9\\
&J = 2, \hspace{2mm}d_2 = 5*2 = 10\\
&J = 3, \hspace{2mm}d_3 = 7*1 = 7
\end{align}
As expected, $\sum_J d_J = 27 = 3*3*3$.

\item 
Using the hint, we note that the only scalar formed from three vectors is
\begin{align}
(\boldsymbol{a}\times\boldsymbol{b})\cdot\boldsymbol{c} &= 
    (\boldsymbol{a}\times\boldsymbol{b})_i c_i\\
&= \epsilon^{ijk}a_j b_k c_i\\
&= \epsilon^{ijk}a_i b_j c_k.
\end{align}
Using this fact, the knowledge that $\boldsymbol{J}$ is a vector operator, the only possible combination that results in $J=0$ (a scalar) would have the same symmetry properties, i.e. symmetry under permutation and antisymmetry under any swap. Therefore, 
\begin{align}
    \ket{J=0, m=0}=\boxed{\ket{01\bar{1}}+\ket{1\bar{1}0}+\ket{\bar{1}01}-
    \ket{0\bar{1}1}-\ket{\bar{1}10}-\ket{10\bar{1}}}.
\end{align}

For the state $J=0$, we need
\begin{multline}
        \ket{j=0,m=0}=\frac{1}{\sqrt{2}}\left(\ket{j_{1+2}=1,m_{1+2}=-1}\otimes \ket{j_3=1,m_3=1}\right.\\\left.-\ket{j_{1+2}=1,m_{1+2}=1}\otimes \ket{j_3=1,m_3=-1}\right)
\end{multline}
but, in turn, we have expressions for the two $j_{1+2}$ states:
\begin{multline}
        \ket{j_{1+2}=1,m_{1+2}=1}=\frac{1}{\sqrt{2}}\left(\ket{j_1=1,m_1=0}\otimes\ket{j_2=1,m_1=1}\right.\\\left.-\ket{j_1=1,m_1=1}\otimes\ket{j_2=1,m_2=0}\right)
\end{multline}
and 
\begin{multline}
        \ket{j_{1+2}=1,m_{1+2}=-1}=\frac{1}{\sqrt{2}}\left(\ket{j_1=1,m_1=0}\otimes\ket{j_2=1,m_2=-1}\right.\\\left.-\ket{j_1=1,m_1=-1}\otimes\ket{j_2=1,m_2=0}\right)
\end{multline}

%%%%%%%%%%%%%%%%%%%%%%%%%%%%%%%%%%%%%%%%%%%%%%%%%%%%%%%%%%%%%%%%%%%%%%%%%%%

\end{enumerate}
    \item[] \textbf{G\&Y: 3.9}
    
\begin{align}
H = -\vec{\mu}\cdot\vec{B}, \hspace{2mm} \vec{\mu}=g\vec{s}
\end{align}
\begin{align}
\frac{d}{dt}s_i &= \frac{i}{\hbar}[s_i, \mu_j B_j]= \frac{i}{\hbar}[s_i,s_j]gB_j\\ &= \frac{i}{\hbar}i\epsilon^{ijk}s_k B_j g = -\frac{g}{\hbar}(\vec{B}\times
    \vec{s})_i\\
&= \frac{g}{\hbar}(\vec{s}\times\vec{B})_i.
\end{align}
Therefore the component of $\vec{s}$ parallel to $\vec{B}$ doesn't change, and the other two precess with $\omega = -\frac{gB}{\hbar}$, meaning
\begin{align}
\theta = -\frac{gBt}{\hbar},
\end{align}
and an angle of $2\pi$ is reached at time $t=2\pi\hbar/gB$. For the experimental setup, since the fields are anti-parallel, we want to rotate the state through $\pi$ radians. If this is done correctly, There will be destructive interference in the center for half-integer spins, and constructive interference in the center for integer spins. For the neutron to rotate $2\pi$, 
\begin{align}
\theta&=2\pi=-\frac{g_nBt}{\hbar}\\
t&=-\frac{2\pi\hbar}{g_nB}.
\end{align}
Assuming the neutrons are moving classically, 
\begin{align}
\frac{1}{2}mv^2 &= \frac{3}{2}kT,\\
v &= \sqrt{\frac{3kT}{m}},\\
l &= \frac{-2\pi\hbar}{g_nB}\sqrt{\frac{m}{3kT}}\\
&= \frac{-2\pi(1.05\cdot10^{-34}Js)}{.1T(-9.66\cdot10^{-27} 
    JT^{-1})}\sqrt{\frac{1.68 \cdot 10^{-27} kg}{3\cdot1.38
    \cdot10^{-23}\cdot10J}}\\
&= \boxed{4.37\cdot10^{-9}\text{m}}\hspace{.5mm}
\end{align}

%%%%%%%%%%%%%%%%%%%%%%%%%%%%%%%%%%%%%%%%%%%%%%%%%%%%%%%%%%%%%%%%%%%%%%%%%%%

\item[] \textbf{S: 3.15}
    
$J_{\pm}=J_x\pm iJ_y.$
\begin{enumerate}[(a)]
\item 
\begin{align}
J^2 &= J_x^2+J_y^2+J_z^2 = J_z^2+J_xJ_x-iJ_xJ_y+iJ_xJ_y+J_yJ_y\\
&= J_z^2 +(Jx+iJ_y)(J_x-iJ_y)+i[J_x,J_y] = \boxed{J_z^2 + J_+J_-+J_z}
\end{align}

\item 
Using the fact that $J_-^\dag = J_+$,
\begin{align}
|c_-|^2 &= c_-^\ast c_- = \bra{j,m-1} c_-^\ast c_-\ket{j,m-1} = 
    \bra{jm}J_+J_-\ket{jm} \\ 
&= \bra{jm}J^2 - J_x^2 + J_z\ket{jm} = j(j+1) - m(m-1)\\
&\implies \boxed{c_- = \sqrt{j(j+1) - m(m-1)}}.
\end{align}
\end{enumerate}

%%%%%%%%%%%%%%%%%%%%%%%%%%%%%%%%%%%%%%%%%%%%%%%%%%%%%%%%%%%%%%%%%%%%%%%%%

    \item[] \textbf{S: 3.24}
    
Start with the $j=2$ states:
\begin{align}
\ket{22} &= \ket{++}\\
c_-\ket{21} &= J_{-,1}\ket{++} + J_{-,2}\ket{++}\\
\ket{21} &= \frac{\sqrt{2-0}}{\sqrt{6-2}}(\ket{+0} + \ket{0+})\\
&= \frac{1}{\sqrt{2}}(\ket{+0} + \ket{0+})\\
c_-\ket{20} &= \frac{\sqrt{2-0}}{\sqrt{6-0}}\frac{1}{\sqrt{2}}(\ket{+-}+\ket{-+}+
    \ket{00}+\ket{00})\\
\ket{20}&= \frac{1}{\sqrt{6}}(\ket{+-}+\ket{-+}) + \sqrt{\frac{2}{3}}\ket{00}.
\end{align}
Restarting from the bottom,
\begin{align}
\ket{2,-2} &= \ket{--}\\
\ket{2,-1} &= \frac{\sqrt{2-0}}{\sqrt{6-2}}(\ket{-0} + \ket{0-})\\
&= \frac{1}{\sqrt{2}}(\ket{+0} + \ket{0+}).
\end{align}
We now know that $\ket{11}$ is orthogonal to $\ket{21}$ and made of states with $m=1$. This means
\begin{align}
\ket{11} &= \frac{1}{\sqrt{2}}(\ket{+0}-\ket{0+})\\
\ket{10} &= \frac{\sqrt{2-0}}{\sqrt{2-0}}\frac{1}{\sqrt{2}}
    (\ket{+-}+\ket{00}-\ket{00}-\ket{-+})\\
&= \frac{1}{\sqrt{2}}(\ket{+-}-\ket{-+})\\
\ket{1,-1} &= \frac{1}{\sqrt{2}}(\ket{0-}-\ket{-0}).
\end{align}
Although there are three basis states for $m=0$, $\ket{00}$ has to be orthogonal to both $\ket{20}$ and $\ket{10}$. From $\ket{10}$, it must be symmetric in $\ket{+-}$ and $\ket{-+}$, so
\begin{align}
\ket{00} &= a(\ket{+-}+\ket{-+}) + b\ket{00}\\
\bra{20}\ket{00} &= \frac{1}{\sqrt{6}}a\cdot2+\sqrt{\frac{2}{3}}b = 
    \sqrt{\frac{2}{3}}a+\sqrt{\frac{2}{3}}b=0\\
\ket{00} &= \frac{1}{\sqrt{3}}(\ket{+-}+\ket{-+}) - \frac{1}{\sqrt{3}}\ket{00}
\end{align}

%%%%%%%%%%%%%%%%%%%%%%%%%%%%%%%%%%%%%%%%%%%%%%%%%%%%%%%%%%%%%%%%%%%%%%

    \item[] \textbf{S: 5.1}
    \begin{enumerate}
\item 
\begin{align}
    \Delta_0^{(1)}&=\braket{n^{(0)}|\lambda H_1|n^{0}}\\&=b\braket{n^{(0)}|x|n^{0}}
\end{align}
Using the result
\begin{equation}
    \braket{n^{(0)}|x|m^{(0)}}=\sqrt{\frac{\hbar}{2m\omega}}\sqrt{m+1}\delta_{n,m+1}+\sqrt{m}\delta_{n,m-1}
\end{equation}
we see that $\Delta_0^{(1)}=0$. The second order expression for $\Delta_{0}$ is given by
\begin{align}
    \Delta_0^{(2)}&=b^2\sum_{k\neq 0}\frac{|\braket{0|x|k}|^2}{E_0^{(0)}-E_k^{(0)}}\\&=\frac{\hbar b^2}{2m\omega}\frac{\sqrt{1}}{E_0^{(0)}-E_1^{(0)}}\\&=\boxed{-\frac{b^2}{2m\omega^2}}
\end{align}
\item 
The Hamiltonian is given by
\begin{align}
    H&=\frac{p^2}{2m}+\frac{m\omega^2x^2}{2}+bx\\&=\frac{p^2}{2m}+\frac{m\omega^2}{2}\left(x+\frac{b}{m\omega^2}\right)^2-\frac{b^2}{2m\omega^2}
\end{align}
If we define the operator $y=x+\frac{b}{m\omega^2}$, it is clear that $[y,p]=i\hbar$, so that the energy levels for the harmonic oscillator are given by
\begin{equation}
    E_n=\hbar\omega\left(n+\frac{1}{2}\right)-\frac{b^2}{2m\omega^2}
\end{equation}
\end{enumerate}

%%%%%%%%%%%%%%%%%%%%%%%%%%%%%%%%%%%%%%%%%%%%%%%%%%%%%%%%%%%%%%%%%%%%%%%%%%%%

    \item[] \textbf{S: 5.4}
    \begin{enumerate}
\item 
\begin{align}
    H_0&=H_x+H_y\\&=\left(\frac{p_x^2}{2m}+\frac{m\omega^2x^2}{2}\right)+\left(\frac{p_y^2}{2m}+\frac{m\omega^2y^2}{2}\right)
\end{align}
It is clear that the energy eigenstates are given by
\begin{equation}
    E_{n,m}=\hbar\omega\left(1+n+m\right)
\end{equation}
The energies of the three lowest-lying sates are $E_{0,0}=\hbar \omega$, $E_{1,0}=E_{0,1}=2\hbar\omega$. There are indeed degeneracies for this Hamiltonian.

\item 
Let 
\begin{equation}
        V=\delta m \omega^2 xy
\end{equation}

The lowest energy state is non-degenerate. Therefore, the $0th$ order eigenstate corresponding to the ground state is given by
\begin{align}
        \ket{n=0,m=0;\lambda^0}&=\ket{0;x}\otimes \ket{0;y}\\&=\int dx dy dz \ket{x,y,z}\left(\frac{\alpha}{\hbar}\right)^{\frac{1}{2}}e^{-\frac{\alpha x^2}{2}}e^{-\frac{\alpha y^2}{2}},\hspace{0.5cm}\alpha=\frac{m\omega}{\hbar}
\end{align}
and 
\begin{align}
        \Delta(n=0,m=0;\lambda^0)&=\braket{n=0,m=0;\lambda^0|V|n=0,m=0;\lambda^0}\\&=\delta m \omega^2 \left(\bra{0;x}\otimes \bra{0;y}\right)xy\left(\ket{0;x}\otimes \ket{0;y}\right)\\&=\delta m \omega^2 \braket{0;x|x|0;x}\braket{0;y|y|0;y}
\end{align}
But we note that the ground-state for the simple harmonic oscillator is invariant under parity inversion, so the expectation values for $x$ and $y$ are $0$. Thus
\begin{equation}
        \Delta(n=0,m=0;\lambda^1)=0
\end{equation}

Next, we consider the degenerate eigenstates. Let
\begin{equation}
        \ket{1}=\ket{n=1,m=0;\lambda^0}=\ket{1;x}\otimes\ket{0;y} 
\end{equation}
and 
\begin{equation}
        \ket{2}=\ket{n=0,m=1;\lambda^0}=\ket{0;x}\otimes\ket{1;y} 
\end{equation}
Then:
\begin{equation}
        \braket{1|V|1}=\delta m\omega^2 \braket{1;x|x|1;x}\braket{0;y|y|0;y}=0
\end{equation}
\begin{equation}
        \braket{2|V|2}=\delta m\omega^2 \braket{0;x|x|1;0}\braket{1;y|y|1;y}=0
\end{equation}
\begin{equation}
        \braket{1|V|2}=\delta m\omega^2 \braket{1;x|x|0;x}\braket{0;y|y|1;y}=\left(\delta m \omega^2\right)\left( \frac{\hbar }{2m\omega}\right)=\frac{\delta \hbar \omega}{2} 
\end{equation}
and 
\begin{equation}
        \braket{2|V|1}=\delta m\omega^2 \braket{0;x|x|1;x}\braket{1;y|y|0;y}=\frac{\delta \hbar \omega}{2}
\end{equation}

Thus, we want to diagonalize the matrix 
\begin{equation}
        V'=\frac{\delta \hbar \omega}{2}\sigma_x\implies E_{\pm}=\delta\frac{\hbar \omega}{2},\chi_{\pm}=\frac{1}{\sqrt{2}}\left(\begin{array}{c}1\\\pm 1\end{array}\right)
\end{equation}

Therefore, the $0th$ order energy eigenstates for the perturbed Hamiltonian in the degenerate subspace are
\begin{equation}
        \ket{1;D;\lambda^0}=\frac{1}{\sqrt{2}}\left(\ket{1}+\ket{2}\right)=\frac{1}{\sqrt{2}}\left(\ket{1;x}\otimes \ket{0;y}+\ket{0;x}\otimes \ket{1;y}\right)
\end{equation}
and 
\begin{equation}
        \ket{2;D;\lambda^0}=\frac{1}{\sqrt{2}}\left(\ket{1}-\ket{2}\right)=\frac{1}{\sqrt{2}}\left(\ket{1;x}\otimes \ket{0;y}-\ket{0;x}\otimes \ket{1;y}\right)
\end{equation}
The corresponding first order corrections to the energies are
\begin{equation}
        \Delta(1;D;\lambda^1)=\delta \frac{\hbar \omega}{2}
\end{equation}
and 
\begin{equation}
        \Delta(2;D;\lambda^1)=-\delta \frac{\hbar \omega}{2}
\end{equation}
\item 
Transform into new coordinates
\begin{align}
u=\frac{x+y}{\sqrt{2}},\hspace{2mm} v=\frac{x-y}{\sqrt{2}}.
\end{align}
Then, the Hamiltonian is given by
\begin{align}
H&=\frac{p_x^2+p_y^2}{2m}+\frac{m\omega^2}{2}(x^2+y^2)+\delta m\omega^2xy\\
&= \frac{p_u^2+p_v^2}{2m} + \frac{m\omega^2}{2} (u^2+v^2) + \delta m\omega^2
    \frac{u^2-v^2}{2}\\
&= \frac{p_u^2+p_v^2}{2m} + \frac{m\omega^2}{2} \left((1 + \delta)u^2
    +(1-\delta)v^2\right).
\end{align}
This is the Hamiltonian for an anisotropic harmonic oscillator with 
\begin{align}
\omega_u^2&=(1 + \delta)\omega^2\\
\omega_v^2&=(1 - \delta)\omega^2.
\end{align}
Eigenstates are then of the form $\ket{n_un_v}$, with energies 
\begin{align}
E_{n_un_v} &= \hbar\omega\left(n_u\sqrt{1 + \delta}+ 
    n_v\sqrt{1 - \delta} +1\right).
\end{align}
This agrees with perturbation theory to first order.
\end{enumerate}
    
%%%%%%%%%%%%%%%%%%%%%%%%%%%%%%%%%%%%%%%%%%%%%%%%%%%%%%%%%%%%%%%%%%%%%%%%%%
     
\item[] \textbf{S: 5.11}
\begin{align}
    \mathcal{H} = \begin{bmatrix}
        E_1^0 & \lambda\Delta\\
        \lambda\Delta & E_2^0
    \end{bmatrix}
\end{align}
\begin{enumerate}
\item 
\begin{align}
    det(\mathcal{H}-\mu\mathbb{1}) &= \begin{vmatrix}
        E_1^0-\mu & \lambda\Delta\\
        \lambda\Delta & E_2^0-\mu
    \end{vmatrix}\\
    &= (E_1^0-\mu)(E_2^0-\mu) -(\lambda\Delta)^2\\
    &= \mu^2 -(E_1^0+E_2^0)\mu +E_1^0E_2^0-\lambda\Delta^2\\
    \implies \Aboxed{\mu =E'&= \frac{E_1^0+E_2^0}{2}\pm \frac{\sqrt{(E_1^0-
    E_2^0)^2-(2\lambda\Delta)^2}}{2}}\hspace{.5mm}.
\end{align}
The vectors are then given by
\begin{align}
    \ket{1} &= \begin{bmatrix}
        \frac{E_2-E_1}{2}- \frac{\sqrt{(E_1^0-E_2^0)^2-(2\lambda\Delta)^2}}{2}\\
        \lambda\Delta
    \end{bmatrix},\\
    \ket{2} &= \begin{bmatrix}
        \frac{E_1-E_2}{2}- \frac{\sqrt{(E_1^0-E_2^0)^2-(2\lambda\Delta)^2}}{2}\\
        \lambda\Delta
    \end{bmatrix},
\end{align}
up to an overall factor.
\item
Using perturbation theory,
\begin{align}
E_1^{(2)} &= E_1^0 + \lambda\braket{\psi_1^0|H'|\psi_1^0} + \lambda^2
    \frac{|\braket{\psi_1^0|H'|\psi_2^0}|^2}{E_1^0-E_2^0}\\
&= \boxed{E_1^0 + \lambda^2\frac{\Delta^2}{E_1^0-E_2^0}}\\
E_2^{(2)} &= \boxed{E_2^0 + \lambda^2\frac{\Delta^2}{E_2^0-E_1^0}}
    \hspace{.5mm}.
\end{align}
As expected, this is the second order approximation to $\mu$ above. The states are also approximations to the above.
\begin{align}
\ket{1} &= \ket{1^0} + \lambda\frac{\Delta}{E_1^0-E_2^0}\ket{2^0}\\
&= \begin{bmatrix}
    1\\ \frac{\lambda\Delta}{E_1^0-E_2^0}\end{bmatrix}, \\
\ket{2} &= \begin{bmatrix}
    \frac{\lambda\Delta}{E_2^0-E_1^0}\\1\end{bmatrix}. \\
\end{align}
\item
Assuming $E_1^0=E_2^0$ and using degenerate perturbation theory,
\begin{align}
H_{eff} &= \lambda\begin{bmatrix}0&\Delta\\ \Delta&0\end{bmatrix} + 0\\
\implies E'^2 &= (\lambda\Delta)^2\\
\implies E &= E_1^0 \pm \lambda\Delta.
\end{align}
The eigenstates are just
\begin{align}
\ket{1} &= \frac{1}{\sqrt{2}}\begin{bmatrix}1\\-1\end{bmatrix}\\
\ket{2} &= \frac{1}{\sqrt{2}}\begin{bmatrix}1\\1\end{bmatrix}
\end{align}

\end{enumerate}    

%%%%%%%%%%%%%%%%%%%%%%%%%%%%%%%%%%%%%%%%%%%%%%%%%%%%%%%%%%%%%%%%%%%%%%%
    
    \item[] \textbf{S: 5.12}
    
We have
\begin{equation}
    H=\left(\begin{array}{ccc}E_1&0&0\\0&E_1&0\\0&0&E_2\end{array}\right)+\left(\begin{array}{ccc}0&0&a\\0&0&b\\a^*&b^*&0\end{array}\right)
\end{equation}
Thus, the unperturbed eigenstates are 
\begin{equation}
        \psi_{1}=\left(\begin{array}{c}1\\0\\0\end{array}\right),\psi_2=\left(\begin{array}{c}0\\1\\0\end{array}\right),\psi_3=\left(\begin{array}{c}0\\0\\1\end{array}\right)
\end{equation}
the first two of which are degenerate. 

Because the perturbation's diagonal elements are $0$, the first order correction to the energy eigenvalues is $0$. Now, the second order non-degenerate approximation to the perturbed eigenvalues is given by
\begin{align}
        E_a&=E_\alpha+\sum_{\beta\neq \alpha}\frac{|\braket{\alpha|H_1|\beta}|^2}{E_\alpha-E_\beta}
\end{align}
Thus, for $E_{1,1}$ we have
\begin{equation}
        E_{1,1}=E_1+\frac{0}{E_1-E_1}+\frac{|a|^2}{E_1-E_2}=E_1+\frac{|a|^2}{E_1-E_2}
\end{equation}
For $E_{1,2}$ we have
\begin{equation}
        E_{1,2}=E_1+\frac{|b|^2}{E_1-E_2}
\end{equation}
and for $E_{2}$ we have
\begin{equation}
        E_2=E_2+\frac{|a|^2+|b|^2}{E_2-E_1}
\end{equation}

Next, Sakurai asks us to diagonalize $H$ to find the exact eigenvalues:

\begin{equation}
        \left|\begin{array}{ccc}E_1-\lambda &0&a\\0&E_1-\lambda&b\\a^*&b^*&E_2-\lambda\end{array}\right|=(E_1-\lambda)((E_1-\lambda)(E_2-\lambda)-|a|^2-|b|^2)=0
\end{equation}
So we conclude that the eigenvalues are given by
\begin{equation}
        \lambda=E_1,\lambda=\frac{E_1+E_2\pm \sqrt{(E_1-E_2)^2+4|a|^2+4|b|^2}}{2}
\end{equation}

Expanding the latter two as a taylor series, we have
\begin{align}
        \lambda_{2,3}&=\frac{(E_1+E_2)\pm \left((E_1-E_2)+\frac{2|a|^2+2|b|^2}{E_1-E_2}\right)}{2}\\&=E_1+\frac{|a|^2+|b|^2}{E_1-E_2},E_2+\frac{|a|^2+|b|^2}{E_2-E_1}
\end{align}

Finally, we use the second order degenerate perturbation theory. We simply need to find the eigenvalues for the effective Hamiltonian given by
\begin{equation}
        H_{eff}=\left(\begin{array}{cc}0&0\\0&0\end{array}\right)+\frac{1}{E_1-E_2}\left(\begin{array}{cc}|a|^2&ab^*\\ba^*&|b|^2\end{array}\right)
\end{equation}

Define $\mu=(E_1-E_2)\lambda$. Then, if $\lambda$ are the eigenvalues of $H_{eff}$, $\mu$ satisfies the equation
\begin{equation}
        \left|\begin{array}{cc}|a|^2-\mu&ab^*\\ba^*&|b|^2-\mu\end{array}\right|=\mu^2-(|a|^2+|b|^2)\mu+|a|^2|b|^2-|a|^2|b|^2=0
\end{equation}

Thus,
\begin{equation}
        \mu=0, |a|^2+|b|^2
\end{equation}

Therefore, the perturbed eigenvalues to second order are
\begin{equation}
        E_{1,1}=E_1,E_{1,2}=E_1+\frac{|a|^2+|b|^2}{E_1-E_2}
\end{equation}
We notice that the second order perturbation for $E_2$ is correctly given by the non-degenerate theory, but that the corrections for $E_{1,1}$ and $E_{1,2}$ are wrong. However, the second order degenerate perturbation theory does the job.

%%%%%%%%%%%%%%%%%%%%%%%%%%%%%%%%%%%%%%%%%%%%%%%%%%%%%%%%%%%%%%%%%%%%%%%%%    
    
\item[] \textbf{S: 5.16}
\begin{enumerate}
\item 
For $s$ states, $l=0$. Using the radial Shr\"{o}dinger equation,
\begin{align}
\frac{d^2 u(r)}{dr^2} = \frac{2m(E-V)}{\hbar^2}u(r).
\end{align}
I'm sure there is some integration by parts that can be done that gives the equation in the book but we couldn't find it.
\item
For the hydrogen atom,
\begin{align}
V(r) &= -\frac{e^2}{4\pi}\frac{1}{r}\\
V'(r) &= \frac{e^2}{4\pi r^2}\\
\psi_{100} &= Y_{00}R_{10}\\
&= \sqrt{\frac{1}{4\pi}}\cdot 2a^{-3/2}e^{-r/a}\\
|\psi_{100}|^2 &= \frac{1}{4\pi}\cdot 4a^{-3}e^{2r/a}\\
|\psi(0)|^2 &= \boxed{\frac{a^{-3}}{\pi}}\\
\frac{m}{2\pi\hbar^2}\braket{V'(r)} &= \frac{1}{2e^2a}\int d\Omega\int_0^\infty 
V'(r)|\psi(r)|^2r^2dr\\
&= \frac{1}{2e^2a}\frac{e^2}{4\pi}\cdot 4\pi \frac{a^{-3}}{\pi}
\int_0^\infty e^{-2r/a}dr\\
&= \frac{a^{-4}}{2\pi}2a\\
&= \boxed{\frac{a^{-3}}{\pi}}\hspace{.5mm}.
\end{align}
Other calculations will look similar.
\end{enumerate}
    
%%%%%%%%%%%%%%%%%%%%%%%%%%%%%%%%%%%%%%%%%%%%%%%%%%%%%%%%%%%%%%%%%%%%%%%%%%
    
\item[] \textbf{S: 5.22}
\begin{align}
    H = \frac{p^2}{2m}+\frac{m\omega_0 x^2}{2}, V(t) = F_0x\cos\omega t.
\end{align}
The unperturbed ket is given by
\begin{align}
    \ket{\psi_0(t)}_I = \ket{0}.
\end{align}
Using time-dependent perturbation theory,
\begin{align}
\ket{\psi(t)}_I &= \sum_n c_n\ket{n}\\
\ket{\psi(t)}_I &= \sum_n c_n^{(0)}\ket{n} + \sum_nc_n^{(1)}\ket{n}\\
&=\sum_n\delta_{n0}\ket{n} - \sum_n \frac{i}{\hbar} \int_0^t
    \braket{n|V_I(t')|0}dt'\\
&=\ket{0}-\frac{i}{\hbar}\sum_n \int_0^t e^{i\omega_{n0}t'}
    \braket{n|V(t')|0}\ket{n}dt'\\
&=\ket{0}-\frac{i}{\hbar}\sum_n \int_0^t e^{i\omega_{n0}t'}F_0
    \braket{n|x|0}\ket{n}\cos\omega t'dt'\\
&=\ket{0}-\frac{i}{\hbar} \int_0^t e^{i\omega_{10}t'}F_0
    \braket{1|x|0}\ket{1}\cos\omega t'dt'\\
&=\ket{0}-\frac{iF_0}{\hbar}\sqrt{\frac{\hbar}{2m\omega_0}}\int_0^t
    e^{i\omega_{10}t'}\cos\omega t'dt'\ket{1}\\
&=\ket{0}-iF_0\sqrt{\frac{1}{2m\omega_0\hbar}}\int_0^t\frac{
    e^{i(\omega_0+\omega)t'}+e^{i(\omega_0-\omega)t'}}{2} t'dt'\ket{1}\\
&=\ket{0}-\frac{iF_0}{2}\sqrt{\frac{1}{2m\omega_0\hbar}}\left(\frac{
    e^{i(\omega_0+\omega)t'}}{\omega_0+\omega}+\frac{
    e^{i(\omega_0-\omega)t'}}{\omega_0-\omega}\right)\ket{1}\\
&=\ket{0}-iF_0\sqrt{\frac{1}{2m\omega_0\hbar}}e^{i\omega_0t}\left(\frac{
    2\omega_0\cos\omega t-2i\omega\sin{\omega t}}
    {\omega_0^2-\omega^2}\right)\ket{1}.
\end{align}
To find the expectation of $x$,
\begin{align}
\braket{\psi(t)|x|\psi(t)} &= iF_0\sqrt{\frac{1}{\dots}} e^{-i\omega_0t}
    \left(\frac{\ +\ }{-}\right)\braket{1|x|0} + c.c\braket{0|x|1}\\
&= \boxed{\frac{F_0}{m\omega_0}\left(\frac{\omega_0\cos\omega t\sin\omega_0 t
    -\omega\sin\omega t\cos\omega_0 t}{\omega_0^2-\omega^2}\right)}
\end{align}
    
%%%%%%%%%%%%%%%%%%%%%%%%%%%%%%%%%%%%%%%%%%%%%%%%%%%%%%%%%%%%%%%%%%%%%%%%%%
%% Find 0-0 probability from 1- (0-2) probability 
%%%%%%%%%%%%%%%%%%%%%%%%%%%%%%%%%%%%%%%%%%%%%%%%%%%%%%%%%%%%%%%%%%%%%%%%%%
    
    \item[] \textbf{S: 5.24}\\
    Consider a particle bound in the ground state $\ket{0}$ of a simple harmonic oscillator potential. At time $t=0$ a perturbation of the form $H'(x,t)=Ax^2e^{-t/\tau}$ is turned on. For $c_n(t)=\braket{n|U_I(t)|0}$ for $t>>\tau$, we have, to first order, for $n=0$:
    \begin{align}
            c_0(t)&=1-\frac{i}{\hbar}\int_0^t\braket{0|V(t')|0}dt'\\
            &=1-\frac{i}{\hbar}\int_0^t e^{-t'/\tau}\braket{0|Ax^2|0}dt'\\
            &=1-\frac{iA\tau}{2m\omega}(1-e^{-t/\tau})
    \end{align}
    The transition probability is:
    \begin{align}
            P_{0\rightarrow 0} &= |c_0(t)|^2\\
            &=1+(\frac{A\tau}{2m\omega})^2(1-e^{-t/\tau})^2\\
            &\simeq1+\Big(\frac{A\tau}{2m\omega}\Big)^2\hspace{2cm}
    \end{align}
\begin{align}
\end{align}
For $n\neq0$, we have: 
\begin{align}
            c_n(t)&=-\frac{i}{\hbar}\int_0^te^{i(E^0_n-E^0_0)t'/\hbar}\braket{n|V(t')|0}dt'\\
            &=-\frac{i}{\hbar}\int_0^te^{(i(E^0_n-E^0_0)/\hbar-1/\tau)t'}\braket{n|Ax^2|0}dt'\\
            &=-\frac{iA}{\sqrt{2}m\omega}\delta_{n,2}\int_0^te^{(i(E^0_n-E^0_0)/\hbar-1/\tau)t'}dt'\\
            &=-\frac{iA\tau}{\sqrt{2}m\omega(1-in\tau\omega))}\delta_{n,2}(1-e^{(in\omega-1/\tau)t})\\
            &\simeq-\frac{iA\tau}{\sqrt{2}m\omega(1-in\tau\omega))}\delta_{n,2}\\
\end{align}
Thus, the transition probability is:
\begin{align}
      P_{0\rightarrow 2} &= |c_2(t)|^2\\
            &= \frac{1}{2(1+(n\tau\omega)^2)}\Big(\frac{A\tau}{m\omega}\Big)^2
\end{align}
and
\begin{align}
      P_{0\rightarrow n} &= |c_n(t)|^2\\
            &= 0
\end{align}
otherwise.

%%%%%%%%%%%%%%%%%%%%%%%%%%%%%%%%%%%%%%%%%%%%%%%%%%%%%%%%%%%%%%%%%%%%%%%%%%
%% Simplify                                                               
%%%%%%%%%%%%%%%%%%%%%%%%%%%%%%%%%%%%%%%%%%%%%%%%%%%%%%%%%%%%%%%%%%%%%%%%%%
    
\item[] \textbf{S: 5.25}
\begin{enumerate}
    
\item
Defining $E_2^0-E_1^0\equiv\hbar\omega_{21}$, time dependent perturbation theory gives
\begin{align}
c_n^{(0)} &= \delta_{ni}\\
c_2^{(1)} &= \frac{-i}{\hbar}\int_0^te^{i\omega_{21}t'}V_{21}(t')dt'\\
&= \frac{-i}{\hbar}\lambda\int_0^te^{i\omega_{21}t'}\cos{\omega t'}dt'\\
&= -\frac{\lambda }{\hbar}\frac{e^{i\omega_{21}t}(\omega_{21}\cos(\omega t)
    -i\omega\sin(\omega t))-i\omega_{21}}{(\omega_{21}-\omega)
    (\omega_{21}+\omega)}\\
&= -\frac{\lambda }{\hbar}\frac{[\omega_{21}\cos{(\omega_{21}t)}\cos{(\omega t)}+\omega\sin{(\omega_{21}t)}\sin{(\omega t)}]}{(\omega_{21}-\omega)(\omega_{21}+\omega)}\\
&-i\frac{\lambda}{\hbar}\frac{[\omega_{21}\sin{(\omega_{21}t)}\cos{(\omega t)}+\omega\cos{(\omega_{21}t)}\sin{(\omega t)}-\omega_{21}]
    }{(\omega_{21}-\omega)(\omega_{21}+\omega)}\\
\end{align}
The transition probability is thus given by:
\begin{align}
P_{1\rightarrow 2}&=|c_2(t)|^2\\
&=\frac{\lambda^2}{\hbar^2}\frac{1}{(\omega_{21}-\omega)^2(\omega_{21}+\omega)^2}[(\omega_{21}\cos{(\omega_{21}t)}\cos{(\omega t)}+\omega\sin{(\omega_{21}t)}\sin{(\omega t)})^2\\
&+(\omega_{21}\sin{(\omega_{21}t)}\cos{(\omega t)}+\omega\cos{(\omega_{21}t)}\sin{(\omega t)}-\omega_{21})^2]
\end{align}

\item If $E_1^0-E_2^0$ is close to $\pm\hbar\omega$, i.e. $(\omega_{21}\mp\omega)\simeq0$, the resulting transition probability would go to infinity, invalidating the previous procedure.

\end{enumerate}
%%%%%%%%%%%%%%%%%%%%%%%%%%%%%%%%%%%%%%%%%%%%%%%%%%%%%%%%%%%%%%%%%%%%%%%%%%

\end{enumerate}
\section*{October 20th}
\begin{enumerate}
    \item[] \textbf{S: 2.20}
    
     Some preliminary work (refresher):
    For a simple harmonic oscillator:
    \begin{align}
        H&=\frac{p^2}{2m}+\frac{m\omega^2x^2}{2}\\&=\frac{1}{2m}\left(p+im\omega x\right)\left(p-im\omega x\right)-\frac{im\omega}{2} \left[x,p\right]\\&=\hbar\omega \left(a^\dagger a +\frac{1}{2}\right)
    \end{align}
    where
    \begin{equation}
        a=\frac{1}{\sqrt{2m\hbar \omega}}\left(p-im\omega x\right)
    \end{equation}
    We note
    \begin{equation}
        \left[a,a^\dagger\right]=1
    \end{equation}
    
    Now for the new stuff. Let $a_{\pm}$ be operators with the following commutation properties:
    \begin{equation}
        \left[a_{\pm},a_{\pm}^\dagger\right]=1,\hspace{1cm}\left[a_{\pm},a_{\mp}\right]=0, \hspace{1cm}\left[a_{\pm},a_{\mp}^\dagger\right]=0
    \end{equation}
    Let
    \begin{equation}
        J_{\pm}=\hbar a^{\dagger}_{\pm}a_{\mp},\hspace{1cm}J_z=\frac{\hbar}{2}\left(a_+^\dagger a_+-a_-^\dagger a_-\right),\hspace{1cm}N=a^\dagger_+a_++a_-^\dagger a_-
    \end{equation}
    Finally, let $\delta_{\pm}=1$ for $+$ and $0$ for $-$.
    Then:
    \begin{align}
        \left[J_z,J_{\pm}\right]&=\frac{\hbar^2}{2}\left(\left[a_+^\dagger a_+,a_{\pm}^\dagger a_{\mp}\right]-\left[a_-^\dagger a_-,a_{\pm}^\dagger a_{\mp}\right]\right)\\&=\frac{\hbar^2}{2}\left(\left[a_+^\dagger a_+,a_\pm ^\dagger\right]a_{\mp}\delta_{\pm}+\left[a_+^\dagger a_+,a_\mp \right]a_{\pm}^\dagger\delta_{\mp}-\left[a_-^\dagger a_-,a_{\mp}\right]a_{\pm}^\dagger \delta_{\pm}-\left[a_-^\dagger a_-,a_{\pm}^\dagger\right]a_{\mp}\delta_{\mp}\right)\\&=\frac{\hbar^2}{2}\left(a_+^\dagger a_-\delta_{\pm}-a_+a_-^\dagger\delta_{\mp}-a_-a_+^\dagger\delta_{\pm}-a_-^\dagger a_+\delta_{\mp}\right)\\&=\hbar \left(\hbar a_+^\dagger a_-\delta_{\pm}-\hbar a_-^\dagger a_+\delta_\mp\right)\\&=\pm \hbar J_{\pm}
    \end{align}
    
    Let $J^2$ be defined as $J^2=J_{+}J_-+J_z^2-\hbar J_z$. Then:
    \begin{align}
        \left[J^2,J_z\right]&=\left[J_+J_-+J_z^2-\hbar J_z,J_z\right]\\&=\left[J_+J_-,J_z\right]\\&=J_+\left[J_-,J_z\right]+\left[J_+,J_z\right]J_-\\&=-\hbar J_+J_-+\hbar J_+J_-\\&=-0
    \end{align}
    Finally
    \begin{align}
        J^2&=J_+J_-+J_z^2-\hbar J_z\\&=\hbar^2 a_+^\dagger a_-a_-^\dagger a_++\frac{\hbar^2}{4}\left(a_+^\dagger a_+-a_-^\dagger a_-\right)^2-\frac{\hbar^2}{2}\left(a_+^\dagger a_+-a_-^\dagger a_-\right)\\&=\\&=\frac{\hbar^2}{4}\left(4a_+^\dagger a_+a_-a_-^\dagger+a_+^\dagger a_+a_+^\dagger a_++a_-^\dagger a_-a_-^\dagger a_--2a_+^\dagger a_+a_-^\dagger a_--2a_+^\dagger a_++2a_-^\dagger a_-\right)\\&=\frac{\hbar^2}{4}\left(4a_+^\dagger a_+a_-^\dagger a_-+4a_+^\dagger a_+[a_-,a_-^\dagger]+a_+^\dagger a_+a_+^\dagger a_++a_-^\dagger a_-a_-^\dagger a_--2a_+^\dagger a_+a_-^\dagger a_--2a_+^\dagger a_++2a_-^\dagger a_-\right)\\&=\frac{\hbar^2}{4}\left(2a_+^\dagger a_+a_-^\dagger a_-+a_+^\dagger a_+a_+^\dagger a_++a_-^\dagger a_-a_-^\dagger a_-+2a_+^\dagger a_++2a_-^\dagger a_-\right)\\&=\frac{\hbar^2}{4}N^2+\frac{\hbar^2}{2}N\\&=\frac{\hbar^2}{2}\left(N\left(N+\frac{1}{2}\right)\right)
    \end{align}

%%%%%%%%%%%%%%%%%%%%%%%%%%%%%%%%%%%%%%%%%%%%%%%%%%%%%%%%%%%%%%%%%%%%%%%%

    \item[] \textbf{S: 2.24}
    
    Let's work in the position basis. In that case, the Hamiltonian is given by
    \begin{equation}
        H=-\frac{\hbar^2}{2m}\frac{d^2}{dx^2}-\nu_0\delta(x) 
    \end{equation}
    We want to find $\psi_n(x)$ s.t. 
    \begin{equation}
        H\psi_n=E_n\psi_n
    \end{equation}
    where $E_n<0$ because we are interested only in bound states.
    For $x<0$, $\psi(x)$ obeys
    \begin{equation}
        \frac{d^2}{dx^2}\psi_n(x)-k^2\psi_n(x)=0,\hspace{1cm}k=\sqrt{-\frac{2mE}{\hbar^2}}
    \end{equation}
    And therefore, for $x<0$, $\psi_n(x)$ takes the form
    \begin{equation}
        \psi_n(x)=A_ne^{k_nx}+B_ne^{-k_nx}
    \end{equation}
    Likewise, for $x>0$, $\psi_n(x)$ takes the form
    \begin{equation}
        \psi_n(x)=C_ne^{k_nx}+D_ne^{-k_nx}
    \end{equation}
    For normalization reasons, $B_n=C_n=0$.
    
    Now, since it is clear that the Hamiltonian is invariant under parity inversion, the wave functions we consider can be restricted to those that satisfy $\psi_n(x)=\pm \psi_n(x)$. Therefore, $A_n=\pm D_n$.
    
    Next, we know that $\psi_n(x)$ has to be continuous at $x=0$ and its derivatives of both sides of $x=0$ satisfy
    \begin{equation}
        \frac{d\psi_n}{dx}\biggr\rvert_{x=\epsilon}-\frac{d\psi_n}{dx}\biggr\rvert_{x=-\epsilon}=-\frac{2m\nu_0}{\hbar^2}
    \end{equation}
    Therefore, $A_n=D_n$, and 
    \begin{equation}
        -2k_nA_n=-\frac{2m\nu_0}{\hbar^2}A_n\implies k_n=\frac{m\nu_0}{\hbar^2}
    \end{equation}
    Finally, because of normalization, we have
    \begin{align}
        1&=2A_n^2\int_0^\infty e^{-2kx}dx\\&=\frac{A_n^2}{k_n}\implies A_n=\sqrt{\frac{m\nu_0}{\hbar^2}}
    \end{align}
    Finally, we have
    \begin{equation}
        k_n=\frac{m\nu_0}{\hbar^2}\implies E=-\frac{m\nu_0^2}{2\hbar^2}
    \end{equation}
    \item[] \textbf{S: 2.40}
    
    Since neutrons are spin-$\frac{1}{2}$ particles, their general spin states can be represented as a superposition of $\ket{\uparrow}$ and $\ket{\downarrow}$, or states with definite values of spin along $\hat{B}$. Let's assume that the beam of neutrons is initially polarized in the state
    \begin{equation}
        \ket{\psi(0)}=\alpha\ket{\uparrow}+\beta\ket{\downarrow}
    \end{equation}
    Let's assume that the particles traveling along the upper path do not encounter a magnetic field. Let's assume that the particles moving along the lower encounter a constant magnetic field $\vec{B}$ over a distance $l$. Since the neutron travels with momentum $p_n=\frac{\hbar}{\bar{\lambda}}$, (assuming velocities in the classical range) that translates to a time in the magnetic field
    \begin{equation}
        t=\frac{m_n\bar{\lambda}l}{\hbar }
    \end{equation}
    
    Now, because the Hamiltonian is given by
    \begin{equation}
        H=-\mu \cdot \vec{B}=-\left(\frac{e g_n}{m_nc}\right)\vec{S}\cdot \vec{B}=-\gamma \vec{S}\cdot \hat{B}
    \end{equation}
    The time evolution of a spin state is given by
    \begin{align}
        U(t,0)&=\text{exp}\left(\frac{i\gamma\vec{S}\cdot \vec{B}t}{\hbar}\right)\\&=\text{exp}\left(\frac{i\gamma \vec{\sigma}\cdot \vec{B}t}{2}\right)\\&=\cos\left(\theta\right)-i\sigma \cdot \hat{n}\sin(\theta),\hspace{1cm}\theta=\frac{\gamma|B|t}{2},\hat{n}=\hat{B}
    \end{align}
    Thus
    \begin{align}
        \ket{\psi(t)}&=U(t,0)\ket{\psi(0)}\\&=e^{-i\theta}\alpha\ket{\uparrow}+e^{i\theta}\beta \ket{\downarrow}
    \end{align}
    As we can see, $\ket{\psi(t)}=\ket{\psi(0)}$, which is the condition for maximum constructive interference, when 
    \begin{equation}
        \theta=2\pi n,\hspace{1cm}n=0,\pm 1,\pm2,\ldots
    \end{equation}
    Thus, for the simplest non-trivial case (i.e., $\theta=2\pi$), we solve for $\theta$ and hence $|B|$:
    \begin{align}
        2\pi=\frac{\gamma |B|t}{2}=\frac{1}{2}\left(\frac{e g_n}{m_nc}\right)\left(\frac{m\bar{\lambda}l}{\hbar}\right)|B|
    \end{align}
    Finally:
    \begin{equation}
        \boxed{|B|=\frac{4\pi c\hbar}{eg_n\bar{\lambda}l}}
    \end{equation}
    
%%%%%%%%%%%%%%%%%%%%%%%%%%%%%%%%%%%%%%%%%%%%%%%%%%%%%%%%%%%%%%%%%%%%%%%%%
    
    \item[] \textbf{S: 3.21}
    
\begin{enumerate}[(a)]
\item 
\begin{align}
L_i &= i\epsilon^{ijk}a_j a_k^\dag = i\epsilon^{ijk}\frac{1}{2
    m\omega}(m\omega x_j + ip_j)(m\omega x_k - ip_k)\\
&= \frac{i}{2m\omega}\epsilon^{ijk}[m^2\omega^2x_jx_k+im\omega p_jx_k
    -im\omega x_jp_k+ p_jp_k]\\
&= \boxed{\epsilon^{ijk}x_jp_k}\hspace{.5mm}. \hspace{2mm}\checkmark\\
L_2 &= L_iL_i = \epsilon_{ijk}\epsilon_{ilm}a_ja_k^\dag a_la_m^\dag
    = (\delta_{jl}\delta_{km}-\delta_{jm}\delta_{kl})a_ja_k^\dag a_la_m^\dag\\
&= (a_ja_k^\dag a_ja_k^\dag + a_ja_k^\dag a_ka_j^\dag)
    = (a_ja_j^\dag a_k^\dag a_k + a_k^\dag a_k^\dag a_ja_j)\\
&= ((a_j^\dag a_j +1)a_k^\dag a_k + a_k^\dag a_k^\dag a_ja_j) = 
    \boxed{(N(N +1) + a_k^\dag a_k^\dag a_ja_j)}\hspace{.5mm}. \hspace{2mm}\checkmark
\end{align}
\item 
By acting $L^2$ and $L_z$ on various trial states $\ket{qlm}$ we find, with all equations of the form $\ket{qlm} = \ket{n_xn_yn_z}$ and denoting $-1 = \minus{1}$
\begin{align}
\ket{010} &= \ket{001}\\
\ket{011} &= \frac{1}{\sqrt{2}}(\ket{100}+ i\ket{010})\\
\ket{01\minus{1}} &= \frac{1}{\sqrt{2}}(\ket{100}- i\ket{010}).
\end{align}
\item 
Assuming this problem is actually asking for $\ket{100}$, 
\begin{align}
\ket{100} &= \frac{1}{\sqrt{3}}(\ket{200}+\ket{020}+\ket{002})
\end{align}
\item 
\end{enumerate}
\begin{align}
\ket{020} &= \frac{1}{6}(\ket{200}+\ket{020}-2\ket{002})\\
\ket{021} &= \frac{1}{\sqrt{2}}(\ket{101}+i\ket{011})\\
\ket{02\bar{1}} &= \frac{1}{\sqrt{2}}(\ket{101}-i\ket{011})\\
\ket{022} &= \frac{1}{2}(\ket{200}-\ket{020})+\frac{i}{\sqrt{2}}\ket{110}\\
\ket{02\bar{2}} &= \frac{1}{2}(\ket{200}-\ket{020})-\frac{i}{\sqrt{2}}\ket{110}.
\end{align}

%%%%%%%%%%%%%%%%%%%%%%%%%%%%%%%%%%%%%%%%%%%%%%%%%%%%%%%%%%%%%%%%%%%%%%%%%%

    \item[] \textbf{S: 3.30}
    \begin{enumerate}[(a)]
\item First, we create the Cartesian vector $\vec{W}=\vec{U}\times \vec{V}$. Then, you can create a spherical tensor of rank $1$ out of $\vec{W}$ with the prescription:
\begin{equation}
        T^{(1)}_q=Y_{l=1}^{m=q}(\vec{W})\implies T^{(1)}_0=\sqrt{\frac{3}{4\pi}}W_z,\hspace{2mm} T^{(1)}_{\pm 1}=\sqrt{\frac{3}{4\pi}}\left(\mp \frac{W_x\pm iW_y}{\sqrt{2}}\right)
\end{equation}
We can drop the $\sqrt{\frac{3}{4\pi}}$ factors, since a scalar multiple of a spherical tensor of rank $k$ is still a spherical tensor of rank $k$. Thus, finally, we have
\begin{equation}
        T^{(1)}_0=U_xV_y-U_yV_x,\hspace{1cm}T_{\pm 1}^{(1)}=\mp \frac{\left(U_yV_z-U_zV_y\right)\pm i\left(U_zV_x-U_xV_z\right)}{\sqrt{2}}
\end{equation}
\item 
Let 
\begin{equation}
        X^{(1)}_0=U_z,X^{(1)}_{\pm 1}=\mp \frac{U_x\pm iU_y}{\sqrt{2}}
\end{equation}
and 
\begin{equation}
        Z^{(1)}_0=V_z,Z^{(1)}_{\pm 1}=\mp \frac{Z_x\pm iZ_y}{\sqrt{2}}
\end{equation}
We can now apply Theorem $3.1$ (Sakurai, page 251) to multiply the two rank-1 spherical tensors:
\begin{align}
    T^{(2)}_q=\sum_{q_1}\sum_{q_2}\braket{1 1;q_1 q_2|1 1; 2 q}X_{q_1}^{(1)}Z_{q_2}^{(1)}
\end{align}

For $j=2$, $j_1=j_2=1$, the non-zero Clebsch-Gordan coefficients are:
\begin{equation}
        \braket{1 1; 1 1|1 1; 2 2}=1
\end{equation}
\begin{equation}
        \braket{1 1; 1 0| 1 1; 2 1}=\braket{1 1; 0 1| 1 1; 2 1}=\frac{1}{\sqrt{2}}
\end{equation}
\begin{equation}
        \braket{1 1; 1 -1| 1 1 2 0}=\braket{1 1; 1 1| 1 1 2 0}=\frac{1}{\sqrt{6}}
\end{equation}
\begin{equation}
        \braket{1 1; 0 0| 1 1 2 0}=\sqrt{\frac{2}{3}}
\end{equation}
and the ones corresponding to negative $m$'s. 

And therefore
\begin{equation}
        T^{(2)}_{\pm 2}=X_{\pm 1}^{(1)}Z_{\pm 1}^{(1)}=\frac{1}{2}\left(U_x\pm iU_y\right)\left(V_x\pm iV_y\right)
\end{equation}
\begin{equation}
        T^{(2)}_{\pm 1}=\frac{1}{\sqrt{2}}\left(X_{\pm 1}^{(1)}Z_0^{(1)}+X_0^{(1)}Z_{\pm 1}^{(1)}\right)=\frac{1}{2}\left(\left(U_x\pm iU_y\right)V_z+U_z(V_x\pm iV_y)\right)
\end{equation}
\begin{align}
    T_0^{(2)}&=\frac{1}{\sqrt{6}}\left(X_{-1}Z_1+X_1Z_{-1}\right)+\sqrt{\frac{2}{3}}X_0Z_0\\&=-\frac{1}{2\sqrt{6}}\left((U_x-iU_y)(V_x+iV_y)+(U_x+iU_y)(V_x-iV_y)\right)+\sqrt{\frac{2}{3}}U_zV_z
\end{align}
\end{enumerate}

%%%%%%%%%%%%%%%%%%%%%%%%%%%%%%%%%%%%%%%%%%%%%%%%%%%%%%%%%%%%%%%%%%%%%%%%%%%%

    \item[] \textbf{S: 4.2}
    
    \begin{enumerate}[(a)]
        \item Two generic translations in 3-space always commute (the product of two translations can be thought of as the addition of displacement vectors).
        \item In general, rotations about different axes do not commute (see rectangular prism rotated about $x$ and $y$ axes by $\frac{\pi}{2}$ radians in both orders).
        \item Translation and parity do not commute. Let's say you start at $x=1$ (in 1-D coordinate). Let's say you translate by $1$ unit, which gets you to $x=2$, and then apply the parity operator. You end up at $x=-2$. If you apply the parity operator first, you get to $x=-1$, and then translation gets you to $x=0$. Clearly, the end result is different.
        \item Rotations and parity inversion do commute. The generators of rotations (angular momentum operators) are invariant under parity inversion. 
    \end{enumerate}
    
    %%%%%%%%%%%%%%%%%%%%%%%%%%%%%%%%%%%%%%%%%%%%%%%%%%%%%%%%%%%%%%%%%%%%%
    
    \item[] \textbf{S: 4.3}
        
    Assume $AB+BA=0$ and $A\ket{\psi}=a\ket{\psi}$ and $B\ket{\psi}=b\ket{\psi}$. Applying $B$ to $A\ket{\psi}$, we have $BA\ket{\psi}=ba\ket{\psi}$. Applying $A$ to $B\ket{\psi}$, we have $AB\ket{\psi}=ab\ket{\psi}$. But $AB=-BA$, so we have $ab=-ba=0$. So one of $a$ or $b$ must be $0$.
    
    Let's consider the momentum operator $p$ and the parity operator $\pi$, which satisfy $\{p,\pi\}=0$. The eigenstates of $p$ are of the form $e^{ikx}$ with corresponding eigenvalues $\hbar k$. Meanwhile, the parity operator's eigenvectors satisfy $f(x)=\pm f(-x)$, with eigenvalues $\pm1$. It is clear that the only state that is simultaneously an eigenstate of $p$ and $\pi$ is the one corresponding to $k=0$. 
    
%%%%%%%%%%%%%%%%%%%%%%%%%%%%%%%%%%%%%%%%%%%%%%%%%%%%%%%%%%%%%%%%%%%%%%%%%
    
    \item[] \textbf{S: 4.11}
    
     The Hamiltonian, which can be written as $H=\frac{\vec{p}^2}{2m}+V(\vec{x})$ is invariant under time-reversal. Therefore, its energy eigenfunctions must be real (up to a phase factor independent of $\vec{x}'$). 
    
    Recall the expressions for the angular momenta in spherical coordinates: 
    \begin{equation}
        L_z=\frac{\hbar}{i}\frac{\partial}{\partial \phi}
    \end{equation}
    \begin{equation}
        L_x=-i\hbar \left(-\sin{\phi}\frac{\partial}{\partial \theta}-\cos{\phi}\cot{\theta}\frac{\partial}{\partial \theta}\right)
    \end{equation}
    and 
    \begin{equation}
        L_y=-i\hbar \left(\cos{\phi}\frac{\partial}{\partial \theta}-\sin{\phi}\cot{\theta}\frac{\partial}{\partial \theta}\right)
    \end{equation}
    
    it is quite clear that for each $L_i$, given a state $\psi(r,\theta,\phi)$ which is real valued, the expectation value $\left<L_i\right>$ will be imaginary. The only way an expectation value can be imaginary is if it is equal to $0$. So
    \begin{equation}
        \left<\vec{L}\right>=0
    \end{equation}
    
    Consider an energy eigenstate written as 
    \begin{equation}
        \psi(r,\theta,\phi)=\sum_{l,m}F_{lm}(r)Y_{l}^m(\theta,\phi)
    \end{equation}
    Because 
    \begin{equation}
        \left(Y^m_l(\theta,\phi)\right)^*=(-1)^m Y_l^{-m}(\theta,\phi)
    \end{equation}
    
    and since $\psi^*=\psi$, we have
    \begin{align}
         \psi^*&=\sum_{l,m}F_{lm}^*(-1)^mY_l^{-m}(\theta,\phi)\\&=\sum_{l,m}(-1)^mF^*_{l,-m}Y_l^m(\theta,\phi)\\&=\sum_{l,m}F_{lm}Y_l^m
    \end{align}
    By the orthogonality property of the spherical harmonics, we conclude
    \begin{equation}
        \boxed{F_{lm}(r)=(-1)^m F^*_{l,-m}}
    \end{equation}
    
    \item[] \textbf{S: 4.12}
    
    Let's first construct the three-dimensional spin-1 matrices:
    \begin{equation}
        S_z=\hbar \left(\begin{array}{ccc}1&0&0\\0&0&0\\0&0&-1\end{array}\right)
    \end{equation}
    
    \begin{equation}
        S_+=\hbar \sqrt{2}\left(\begin{array}{ccc}0&1&0\\0&0&1\\0&0&0\end{array}\right)
    \end{equation}
    and 
    \begin{equation}
        S_-=\hbar \sqrt{2}\left(\begin{array}{ccc}0&0&0\\1&0&0\\0&1&0\end{array}\right)
    \end{equation}
    And, from $S_{\pm}=S_x\pm iS_y$, we get
    \begin{equation}
        S_x=\frac{\hbar}{\sqrt{2}}\left(\begin{array}{ccc}0&1&0\\1&0&1\\0&1&0\end{array}\right)
    \end{equation}
    and 
    \begin{equation}
        S_y=\frac{\hbar}{\sqrt{2}}\left(\begin{array}{ccc}0&-i&0\\i&0&-i\\0&i&0\end{array}\right)
    \end{equation}
    Thus
    \begin{equation}
        S_z^2=\hbar^2\left(\begin{array}{ccc}1&0&0\\0&0&0\\0&0&1\end{array}\right),\hspace{0.5cm}S_x^2=\frac{\hbar^2}{2}\left(\begin{array}{ccc}1&0&1\\0&2&0\\1&0&1\end{array}\right),\hspace{0.5cm}S_y^2=\frac{\hbar^2}{2}\left(\begin{array}{ccc}1&0&-1\\0&2&0\\-1&0&1\end{array}\right)
    \end{equation}
    Therefore, the Hamiltonian is
    \begin{align}
        H&=AS_z^2+B(S_x^2-S_y^2)\\&=\hbar^2\left(\begin{array}{ccc}A&0&B\\0&0&0\\B&0&A\end{array}\right)
    \end{align}
    
    Thus, $H$ has eigenvectors and eigenvalues given by
    
    \begin{equation}
        H\left(\begin{array}{c}0\\1\\0\end{array}\right)=0\left(\begin{array}{c}0\\1\\0\end{array}\right),H\left(\begin{array}{c}\frac{1}{\sqrt{2}}\\0\\\frac{1}{\sqrt{2}}\end{array}\right)=(A+B)\left(\begin{array}{c}\frac{1}{\sqrt{2}}\\0\\\frac{1}{\sqrt{2}}\end{array}\right),H\left(\begin{array}{c}\frac{1}{\sqrt{2}}\\0\\-\frac{1}{\sqrt{2}}\end{array}\right)=(A-B)\left(\begin{array}{c}\frac{1}{\sqrt{2}}\\0\\-\frac{1}{\sqrt{2}}\end{array}\right)
    \end{equation}
    The Hamiltonian is invariant under time reversal, because $S_i\to -S_i$ under time reversal, so $S_i^2\to S_i^2$. And because 
    \begin{equation}
        \Theta^{-1}S_z\Theta =-S_z\implies S_z\Theta\ket{m}=-\Theta S_z\ket{m}=-\hbar m\Theta\ket{m}\implies \Theta\ket{m}=e^{i\delta }\ket{-m}
    \end{equation}
    we see
    \begin{align}
        \Theta\left(\begin{array}{c}1\\0\\0\end{array}\right)=e^{i\delta_1}\left(\begin{array}{c}0\\0\\1\end{array}\right),\Theta\left(\begin{array}{c}0\\1\\0\end{array}\right)=e^{i\delta_0}\left(\begin{array}{c}0\\1\\0\end{array}\right),\Theta\left(\begin{array}{c}0\\0\\1\end{array}\right)=e^{i\delta_{-1}}\left(\begin{array}{c}1\\0\\0\end{array}\right)
    \end{align}
    However, $\delta_1$ and $\delta_{-1}$ are related because for integer spins, $\Theta^2=1$. Therefore, $e^{-i\delta_1}e^{i\delta_{-1}}=1\implies e^{i\delta_1}=e^{-\delta_{-1}}$. 
    
    And therefore
    \begin{equation}
        \Theta\psi_0=e^{i\delta_0}\psi_0,\hspace{1cm}\Theta\psi_{A+B}=e^{i\delta_{1}}\psi_{A+B},\hspace{1cm}\Theta\psi_{A-B}=e^{i\delta_1}\psi_{A-B}
    \end{equation}
    And, if we adopt the phase convention $\Theta\ket{j,m}=i^{2m}\ket{j,m}$, then we find
    \begin{equation}
            \Theta\psi_0=\psi_0,\hspace{1cm}\Theta\psi_{A+B}=-\psi_{A-B}
    \end{equation}

\end{enumerate}

\section*{November 4th}
\begin{enumerate}
    \item[] \textbf{S: 4.8}
    
    \begin{enumerate}[(a)]
        \item The problem as stated is absurd. The only freedom one has to play with is the phase. A general state with
        \begin{equation}
            \psi(x)=f(x)+ig(x)
        \end{equation}
        with $f(x)$ and $g(x)$ square integrable, is a valid wave function. However, it is not necessarily the case that 
        \begin{equation}
            \psi(x)=e^{i\delta}h(x)=\cos{\delta}h(x)+i\sin{\delta}h(x)
        \end{equation}
        because $f(x)$ and $g(x)$ need not be scalar multiples of one another. 
        
        Therefore, I think the problem meant to ask us to prove that the \textit{energy eigenfunction} in position space can always be made real. I recreate the proof given in the book.
        
        Because the Hamiltonian is invariant, we have $H\Theta=\Theta H$. Thus
        \begin{equation}
            H\Theta \ket{n}=\Theta H\ket{n}=E_n\Theta \ket{n}
        \end{equation}
        Since $H$ is not degenerate, we conclude
        \begin{equation}
            \Theta \ket{n}=e^{i\delta} \ket{n}\implies \braket{x''|\Theta|n}=e^{i\delta}\braket{x''|n}
        \end{equation}
        
        But also
        \begin{align}
            \braket{x''|\Theta|n}&=\bra{x''}\Theta\left(\int d^3x' \ket{x'}\braket{x'|n} \right)\\&=\bra{x''}\int d^3 x' \ket{x'}\braket{x'|n}^*\\&=\braket{x''|n}^*
        \end{align}
        
        Let $\phi(x)=\braket{x|n}$. Let $\psi(x)=e^{i\delta/2}\phi(x)$. From the results above, we see $e^{i\delta}\phi(x)=\phi^*(x)\implies \psi(x)=\psi(x)^*$, so $\psi(x)$, which differs from $\braket{x|n}$ merely by a phase factor, is real. QED.
        \item
        It is not a problem that $e^{ikx}$, an energy eigenfunction for the free-particle Hamiltonian, cannot be expressed as a real function (up to a phase), because the state $e^{-ikx}$ is a state with the same energy. Hence, the Hamiltonian is degenerate, and the theorem proven above does not hold.

    \end{enumerate}
    
    \item[] \textbf{S: 4.9}
    
    \begin{align}
        \Theta \ket{\alpha}&=\Theta \int d^3p' \ket{\vec{p}}\braket{\vec{p}|\alpha}\\&=\int d^3p' \braket{\vec{p}'|\alpha}^*\Theta \ket{\vec{p'}}\\&=\int d^3 p' \braket{\vec{p}'|\alpha}^*\ket{-\vec{p}'}\\&=\int d^3p'\ket{\vec{p'}}\braket{-\vec{p}'|\alpha}
    \end{align}
    Taking the inner product with $\bra{\vec{p}}$ on the right, we find
    \begin{equation}
        \phi_\alpha(\vec{p}')\equiv \braket{\vec{p}'|\alpha}\rightarrow \braket{-\vec{p}'|\alpha}^*\equiv \phi_{\alpha}(-\vec{p}')^*
    \end{equation}
    
    \item[] \textbf{S: 4.10}
    
     
    Note
    \begin{equation}
        \Theta \ket{j,m}=e^{i\delta_m}\ket{j,-m}
    \end{equation}
    because 
    \begin{equation}
        \Theta \vec{J}\Theta^{-1}=-\vec{J}
    \end{equation}
    Also, because
    \begin{equation}
        \Theta^2\ket{j,m}=(-1)^{2j}\ket{j,m}
    \end{equation}
    we conclude
    \begin{equation}
        e^{-i\delta_m}e^{i\delta_{-m}}=(-1)^{2j}
    \end{equation}
     \begin{enumerate}[(a)]
        \item
        \begin{align}
            \Theta D(R)\ket{j,m}&=D(R)\Theta \ket{j,m} \\&=e^{i\delta_m} D(R) \ket{j,-m}
        \end{align}
        because rotations and motion-reversal commute.
        \item 
        \begin{align}
           \Theta D(R)\ket{j,m}&=\Theta \sum_{m'}\ket{j,m'}\braket{j,m'|D(R)|j,m}\\&=\Theta \sum_{m'}\ket{j,m'}D^{(j)}_{m'm}\\&=\sum_{m'}e^{i\delta_{m'}}\ket{j,-m'}D^{(j)*}_{m'm}
        \end{align}
        And so
        \begin{align}
            \braket{j,m'|\Theta D(R)|j,m}&=e^{i\delta_{-m'}}D^{(j)*}_{-m',m}
        \end{align}
        
        Simultaneously,
        \begin{align}
            \Theta D(R)\ket{j,m}&=D(R)\Theta \ket{j,m}\\&=e^{i\delta_m} D(R)\ket{j,-m}\\&=\sum_{m'}e^{i\delta_m}\ket{j,m'}D^{(j)}_{m',-m}
        \end{align}
        And so
        \begin{equation}
            \braket{j,m'|\Theta D(R)|j,m}=e^{i\delta_m}D^{(j)}_{m',-m}
        \end{equation}
        Thus, we get the result
        \begin{equation}
            e^{i\delta_{-m'}}D^{(j)*}_{-m',m}=e^{i\delta_m}D^{(j)}_{m',-m}
        \end{equation}
        Letting $m'\to -m'$, we thus finally get
        \begin{equation}
            D^{(j)^*}_{m',m}=e^{i(\delta_m-\delta_{m'})}D^{(j)}_{-m',-m}
        \end{equation}
        
        We now use the result from part $(c)$ which states that $e^{\delta_m}=i^{2m}$. Thus, the result above becomes
                \begin{equation}
            D^{(j)^*}_{m',m}=i^{2(m-m')}D^{(j)}_{-m',-m}=(-1)^{m-m'}D^{(j)}_{-m',-m}
        \end{equation}
        \item 
        Let us assume that for integer spin, the following holds
        \begin{equation}
            \Theta\ket{lm}=(-1)^m\ket{l,-m}=i^{2m}\ket{j,-m}
        \end{equation}
        We know that, for the phase convention adopted for spherical harmonics, the above holds for orbital momentum eigenstates. 
        
        Furthermore, following Sakurai, we adopt the following convention for the time reversal operator's action on spin-1/2 states:
        \begin{equation}
            \Theta\ket{\uparrow}=i\ket{\downarrow}=i^{2\left(\frac{1}{2}\right)}\ket{\downarrow},\hspace{1cm}\Theta\ket{\downarrow}=i^{-1}\ket{\uparrow}=i^{2\left(-\frac{1}{2}\right)}\ket{\uparrow}
        \end{equation}
        
        We now proceed with a proof by induction. 
        
        Base case: let's add the angular momentum states to get the highest $j$ and $m$ state:
        \begin{equation}
            \ket{j=l+\frac{1}{2},m=l+\frac{1}{2}}=\ket{l,m_l=l}\otimes \ket{\frac{1}{2},\frac{1}{2}}
        \end{equation}
        Thus
        \begin{align}
            \Theta\ket{j=l+\frac{1}{2},m=l+\frac{1}{2}}&=(\Theta\otimes\Theta)\ket{ll}\otimes \ket{\frac{1}{2},\frac{1}{2}}\\&=i^{2\left(l+\frac{1}{2}\right)}\ket{l,-l}\otimes \ket{\frac{1}{2},-\frac{1}{2}}\\&=i^{2m}\ket{j=l+\frac{1}{2},m=-l-\frac{1}{2}}
        \end{align}
        
        Inductive step: assume that 
        \begin{equation}
                \Theta\ket{j=l+\frac{1}{2},m}=i^{2m}\ket{j,m}
        \end{equation}
        Then:
        \begin{align}
                \Theta \ket{j=l+\frac{1}{2},m-1}&=\frac{1}{\sqrt{j(j+1)-m(m-1)}}\Theta\left(J_-\ket{j=l+\frac{1}{2},m}\right)\\&=\frac{1}{\sqrt{j(j+1)-m(m-1)}}\Theta\left((J_x+iJ_y)\ket{j=l+\frac{1}{2},m}\right)\\&=-\frac{1}{\sqrt{j(j+1)-m(m-1)}}J_+\Theta\ket{j=l+\frac{1}{2},m}\\&=\frac{i^{2(m+1)}}{\sqrt{j(j+1)-m(m-1)}}J_+\ket{j=l+\frac{1}{2},-m}\\&=i^{2(m+1)}\frac{\sqrt{j(j+1)-(-m)(-m+1)}}{\sqrt{j(j+1)-m(m-1)}}\ket{j=l+\frac{1}{2},-(m-1)}\\&=i^{2(m-1)}i^4\ket{j=l+\frac{1}{2},-(m-1)}\\&=i^{2(m-1)}\ket{j=l+\frac{1}{2},-(m-1)}
        \end{align}
        Thus, by the principle of mathematical induction, for any $j=l+\frac{1}{2}$ and $-j\leq m\leq j$, we conclude that $\Theta\ket{j,m}=i^{2m}\ket{j,-m}$. 
        
        Since we already knew the phase for all integer angular momentum states, and we now have shown that the same rule applies to all half-integer angular momentum states, we conclude that quite generally
        \begin{equation}
                \Theta\ket{j,m}=i^{2m}\ket{j,-m}
        \end{equation}
    \end{enumerate}
    \item[] \textbf{S: 5.8}
    
    \begin{enumerate}[(a)]
\item 
Let $U_{\pm 1}=\mp\frac{x\pm iy}{\sqrt{2}}$ and $U_0=z$, thus defining the coordinates of a rank 1 spherical tensor. Then
\begin{equation}
        \braket{n=2,l=1,m=0|x|n=2,l=0,m=0}=0
\end{equation}
because $x=-\frac{U_1+U_-}{\sqrt{2}}$, and by the m-selection rule for spherical tensors, the matrix elements of $U_1$ and $U_{-1}$ separately must be $0$.
\item The Wigner-Eckart theorem does not let us conclude that the matrix element is $0$. Neither does Parity. From the time-reversal operator, we see that the matrix element is imaginary:
\begin{equation}
    \ket{\alpha}=\ket{n=2,l=1,m=0}\rightarrow \ket{\tilde{\alpha}}=i^0\ket{n=2,l=1,m=0}=\ket{\alpha}
\end{equation}
and 
\begin{equation}
    \ket{\beta}=\ket{n=2,l=0,m=0}\rightarrow \ket{\tilde{\alpha}}=i^0\ket{n=2,l=0,m=0}=\ket{\beta}
\end{equation}
So
\begin{align}
        \braket{\alpha|p_z|\beta}&=\braket{\tilde{\beta}|\Theta p_z\Theta^{-1}|\tilde{\alpha}}\\&=\braket{\beta|(-p_z)|\alpha}\\&=-\ket{\alpha|p_z|\beta}^*
\end{align}
We are forced to evaluate the matrix element by direct computation. In the coordinate basis, we have
\begin{equation}
    p_z=\frac{\hbar}{i}\frac{\partial}{\partial z}=\frac{\hbar}{i}\left[\frac{\partial r}{\partial z}\frac{\partial}{\partial r}+\frac{\partial \theta}{\partial z}\frac{\partial}{\partial \theta}+\frac{\partial \phi}{\partial z}\frac{\partial}{\partial \phi}\right]=\frac{\hbar}{i}\left[\frac{1}{\cos{\theta}}\frac{\partial}{\partial r}-\frac{1}{r\sin{\theta}}\frac{\partial}{\partial \theta}\right]
\end{equation}
We note:
\begin{equation}
    Y_1^0(\theta,\phi)=\sqrt{\frac{3}{4\pi}}\cos{\theta},\hspace{1cm}Y_0^0(\theta)=\frac{1}{\sqrt{4\pi}}
\end{equation}
and 
\begin{equation}
    R_{20}=\frac{2}{(2a_0)^{\frac{3}{2}}}\left(1-\frac{r}{2a_0}\right)e^{-\frac{r}{2a_0}},\hspace{1cm}R_{21}=\frac{1}{\sqrt{3}(2a_0)^{\frac{3}{2}}}\frac{r}{a_0}e^{-\frac{r}{2a_0}}
\end{equation}
Thus
\begin{align}
    \braket{n=2,l=1,m=0|p_z|n=2,l=0,m=0}&=\frac{\sqrt{3}\hbar}{i}\int_0^\infty R_{21}\frac{\partial}{\partial r}\left(R_{20}\right)r^2dr\\&=\text{constant}\times \frac{\hbar}{a_0}
\end{align}
\item 
Let's lower the state $j=9/2$, $m=9/2$, $l=4$, $s=\frac{1}{2}$:
\begin{align}
         &J_-\ket{l=4,s=\frac{1}{2};j=\frac{9}{2}m=\frac{9}{2}}\\&=\hbar \sqrt{\frac{9}{2}\frac{11}{2}-\frac{9}{2}\frac{7}{2}}\ket{ls;j=\frac{9}{2}m=\frac{7}{2}}\\&=3\hbar \ket{ls;jm=\frac{7}{2}}\\&=\left(L_-+S_-\right)\ket{l=4,m_l=4;s=\frac{1}{2},m_s=\frac{1}{2}}\\&=\hbar 2\sqrt{2}\ket{m_l=3;m_s=\frac{1}{2}}+\hbar\ket{m_l=4;m_s=-\frac{1}{2}}
\end{align}
and therefore
\begin{equation}
        \ket{l=4,s=\frac{1}{2}lj=\frac{9}{2},m=\frac{7}{2}}=\frac{2\sqrt{2}}{3}\ket{m_l=3;m_s=\frac{1}{2}}+\frac{1}{3}\ket{m_l=4,m_s=-\frac{1}{2}}
\end{equation}
Evaluating $L_z$ in this state, we find
\begin{equation}
        \braket{L_z}=\hbar \left(3\frac{8}{9}+\frac{4}{9}\right)=\frac{28}{9}\hbar 
\end{equation}
\item 
The singlet and triplet states with $m_s=0$ are
\begin{equation}
        \ket{00}=\frac{1}{\sqrt{2}}\left(\ket{\uparrow \downarrow}-\ket{\downarrow\uparrow}\right),\hspace{1cm}\ket{10}=\frac{1}{\sqrt{2}}\left(\ket{\uparrow \downarrow}+\ket{\downarrow\uparrow}\right)
\end{equation}
where the first spin refers to the electron, and the second to the positron. Therefore, we see
\begin{equation}
        S_z^{(e-)}\ket{\text{singlet}}=\hbar \ket{\text{triplet},m_s=0}
\end{equation}
and 
\begin{equation}
        S_z^{(e+)}\ket{\text{singlet}}=-\hbar \ket{\text{triplet},m_s=0}
\end{equation}
Therefore
\begin{equation}
        \braket{\text{singlet}|S_z^{e-}-S_z^{e+}|\text{triplet},m_s=0}=2\hbar
\end{equation}
\item
\begin{equation}
        \vec{S}_1\cdot \vec{S}_2=\frac{1}{2}\left(\vec{S}^2-\vec{S}_1^2-\vec{S}_2^2\right)
\end{equation}
For an electron, a spin-1/2 particle, $\vec{S}_1^2=\vec{S}_2^2=\hbar^2\left(\frac{1}{2}\right)\left(\frac{1}{2}+1\right)=\frac{3}{4}\hbar^2$. Furthermore, in the ground state of the hydrogen molecule, the two electrons will be in the singlet state, so $\vec{S}^2=0$. Thus, we have
\begin{equation}
        \braket{\vec{S}_1\cdot \vec{S}_2}=-\frac{3}{4}\hbar^2
\end{equation}
\end{enumerate}

%%%%%%%%%%%%%%%%%%%%%%%%%%%%%%%%%%%%%%%%%%%%%%%%%%%%%%%%%%%%%%%%%%%%%%%%%%%

\item[] \textbf{S: 5.17}
\begin{align}
    H = AL^2+BL_z+CL_y
\end{align}
\begin{enumerate}
\item
for $H=AL^2+BL_z$, the energies are $E^0_{lm} = Al(l+l)+Bm$. Assuming $B<<A$, there are no degeneracies. The energies are then
\begin{align}
E'_{lm} &= E^0_{lm} + C\braket{lm|L_y|lm} + C^2\sum_{lm\neq l'm'}
    \frac{|\braket{lm|L_y|l'm'}|^2}{E_{lm}-E_{l'm'}}.
\end{align}
The first order term is 0 by the Wigner-Eckhart theorem. For the second term, $L_y=\frac{1}{2i}(L_+-L_-)$, so
\begin{align}
E_{lm}^{(2)} &= C^2\sum_{lm\neq l'm'}\frac{|\braket{lm|L_y|l'm'}|^2}
    {E_{lm}-E_{l'm'}}\\
&= C^2\frac{|\braket{lm|L_+|lm-1}|^2}{2(E_{lm}-E_{lm-1})} + 
    C^2\frac{|\braket{lm|L_-|lm+1}|^2}{2(E_{lm}-E_{lm+1})}\\
&= C^2\frac{l(l+1)-m(m-1)}{2B}+C^2\frac{l(l+1)-m(m+1)}{-2B}\\
&= C^2\frac{l(l+1)-l(l+1)+m(m+1)-m(m-1)}{2B}\\
&=\frac{C^2m}{B}.
\end{align}
Therefore,
\begin{align}
E_{lm}\approx \boxed{Al(l+l)+Bm+\frac{C^2m}{B}}.
\end{align}
\end{enumerate}
    
%%%%%%%%%%%%%%%%%%%%%%%%%%%%%%%%%%%%%%%%%%%%%%%%%%%%%%%%%%%%%%%%%%%%%%%%%%%%
    
    \item[] \textbf{S: 5.23}
    \begin{enumerate}
        \item 
        \begin{equation}
            H=H_0+V,\hspace{1cm} H=\frac{p^2}{2m}+\frac{m\omega^2 x^2}{2},\hspace{1cm}V=-F_0e^{-\frac{t}{\tau}}x
        \end{equation}
        and
        \begin{equation}
            H_0\ket{n}=\hbar \omega\left(n+\frac{1}{2}\right)\ket{n}
        \end{equation}
        
        Now,
        \begin{equation}
            \ket{\psi,t_0;t}_I=U_I(t,t_0)\ket{0}=\sum_{n}c_n(t)\ket{n}
        \end{equation}
        where 
        \begin{align}
           c_n(t)&=\braket{n|U_I(t,t_0)|0}\\&=c_n^{(0)}(t)+c_n^{(1)}(t)+\ldots\\&=\delta_{n0}-\frac{i}{\hbar}\int_0^t e^{i\omega_{n0}t'}V_{n0}(t')dt'
        \end{align}
        where $\omega_{n0}=(E_n-E_0)/\hbar=n\omega$ and 
        \begin{align}
            V_{n0}(t')=-F_0e^{-\frac{t'}{\tau}}\braket{n|x|0}
        \end{align}
        Given 
        \begin{equation}
            \braket{n|x|m}=\sqrt{\frac{\hbar}{2m\omega_0}}\left(\sqrt{m+1}\delta_{n,m+1}+\sqrt{m}\delta_{n,m-1}\right)
        \end{equation}
        we find
        \begin{equation}
            V_{n0}(t')=-F_0e^{-\frac{t'}{\tau}}\sqrt{\frac{\hbar}{2m\omega_0}}\delta_{n1}
        \end{equation}
        Thus, $c_{n0}=0$ for $n>1$ to first order, and 
        \begin{align}
            c_{10}(t)&=F_0\sqrt{\frac{\hbar}{2m\omega_0}}\frac{i}{\hbar}\int_0^t e^{in\omega t'}e^{-\frac{t'}{\tau}}dt'\\&=i\frac{F_0}{\sqrt{2m\hbar \omega_0}}\frac{1}{in\omega -\tau^{-1}}\left(e^{t (in\omega-\tau^{-1}}-1\right)
        \end{align}
        And so the probability of transitioning is given by
        \begin{align}
            P_{0\to 1}&=\frac{F_0^2}{2m\hbar \omega}\frac{1}{\left|i\omega -\tau^{-1}\right|}\left|e^{t\left(i\omega-\tau^{-1}\right)}-1\right|\\&=\frac{F_0^2}{2m\hbar \omega}\frac{1}{\omega^2+\tau^{-2}}\left(1+e^{-\frac{2t}{\tau}}-2\cos\left(\omega t\right)e^{-\frac{t}{\tau}}\right)
        \end{align}
        
        As $t\to \infty$, we have
        \begin{equation}
            P_{0\to 1}=\frac{F_0^2}{2m\hbar \omega}\frac{1}{\omega^2+\tau^{-2}}
        \end{equation}
        which is independent of time. In retrospect, this result is not too surprising, since as $t\to\infty$, the time-dependent force goes to $0$. Therefore, after a sufficiently long period of time, there's no force to driving transitions, so the probabilities for transitions should level out.
        
        \item To first order, we cannot find higher excited states. Namely, $|c_{n}(t)|^2$ is non-zero only if we include second-order terms and higher. 
    \end{enumerate}
\end{enumerate}
\section*{November 10th}
\begin{enumerate}
    \item[] \textbf{S: 3.31}
    
    \begin{enumerate}[(a)]
\item 
Let 
$T_{\pm 1}=\mp \frac{x\pm iy}{\sqrt{2}}$, and $T_0=z$. $T$ is a spherical tensor of rank $1$. Therefore, by the Wigner-Eckart Theorem
\begin{equation}
        \braket{n'j'm'|T_q|njm}=\braket{j1;mq|j1;j'm'}\frac{\braket{n'j'||T||nj}}{\sqrt{2j+1}}
\end{equation}

The constant in the numerator is independent of $q$, $m'$, and $m$. For ease of notation, call the entire fraction $C(n',j',n,j,T)\equiv C$.
Therefore
\begin{equation}
        \braket{n'j'm'|\mp\frac{x\pm iy}{\sqrt{2}}|njm}=\braket{j1;m,\pm1|j1;j'm'}C
\end{equation}
and 
\begin{equation}
        \braket{n'j'm'|z|njm}=\braket{j1;m0|j1;j'm'}C
\end{equation}
Specifically, if $|j-1|>j'$ or $j+1<j'$, then the matrix element is $0$ (regardless of the value of $q$). Furthermore, if $m+q\neq m'$, the matrix element is $0$. Thus, we see that for different values of $q$, the $T_q$ is non-vanishing for different matrix elements. Thus, there is no matrix element of $T_q$ that is non-zero for all $q$.


\item Let
\begin{equation}
        \psi(\vec{x})=\braket{\vec{x}|\psi}=R_{nl}(r)Y_l^m(\theta,\phi)
\end{equation}
Let $A_{\pm 1}=\mp \frac{x\pm i}{\sqrt{2}}$ and $A_0=z$. We recognize
\begin{equation}
        A_i\ket{\vec{x}}=A_i'\ket{\vec{x}},\hspace{1cm}A_i'\equiv \sqrt{\frac{4\pi}{3}}rY_1^i (\theta,\phi)
\end{equation}

Then:
\begin{align}
    \braket{n'j'm'|A_i|njm}&=\int d\vec{x} \braket{n'j'm'|\vec{x}}\braket{\vec{x}|A_i|njm}\\&=\int r^2 dr \int \sin{\theta} d\theta \int d\phi R^*_{n'l'}Y_{l'}^{m'*}(\theta,\phi)A_i' R_{nl}Y_l^m(\theta,\phi)\\&=\left[\sqrt{\frac{4\pi}{3}}\int r^3 dr R_{n',l'}R_{nl}\right]\int \sin{\theta} d\theta \int d\phi Y_{l'}^{m'*}(\theta,\phi)Y_1^i(\theta,\phi)Y_l^m(\theta,\phi)
\end{align}
Now we recall the result at the bottom of page $231$ in Sakurai, which simplifies integrals involving three spherical harmonics. Our result above becomes
\begin{align}
        \braket{n'j'm'|A_i|njm}&=\left[\sqrt{\frac{4\pi}{3}}\int r^3 dr R_{n',l'}R_{nl}\right]\sqrt{\frac{(3)(2l+1)}{4\pi (2l'+1)}}\braket{1l;00|l1;l'0|}\braket{1l;im|1l;l'm'}\\&=F(n,n',l,l
        ')\braket{1l;im|1l;l'm'}
\end{align}
This mirrors our answer in part $(a)$.
\end{enumerate}

    \item[] \textbf{S: 3.32} 
    \begin{enumerate}[(a)]
\item It is not possible for $xy$, $xz$ and $x^2-y^2$ to be components of a rank 2 spherical tensor because
\begin{align}
        [J_z,xy]&=[J_z,x]y+x[J_z,y]\\&=-i\hbar \left(\epsilon_{132}y^2+\epsilon_{231}x^2\right)\\&=\hbar\left(y^2-x^2\right)\neq \hbar q xy
\end{align}

However, we can express $xy$, $xz$ and $x^2-y^2$ as sums of the different components of a rank-2 spherical tensor. Let 
\begin{equation}
    U_{\pm 1}=\mp\frac{x\pm iy}{\sqrt{2}},U_0=z
\end{equation}
We construct the rank-2 spherical tensor $T^{(2)}$ by combining $U^{(1)}$ with itself, getting the following components for $T$:
\begin{equation}
    T_{\pm 2}=U_{\pm 1}U_{\pm 1}=\frac{1}{2}\left(x\pm iy\right)^2=\frac{1}{2}\left(x^2-y^2\pm 2ixy\right)
\end{equation}
\begin{equation}
    T_{\pm 1}=\sqrt{2}U_{\pm 1}U_0=\mp \left(x\pm iy\right)z=\mp xz-iyz
\end{equation}
and 
\begin{equation}
    T_0=\frac{1}{\sqrt{6}}\left(2U_{+1}U_{-1}+2U_0^2\right)=\frac{1}{\sqrt{6}}\left(2z^2-x^2-y^2\right)
\end{equation}
Thus, we can see that
\begin{equation}
    xy=\frac{T_{+2}-T_{-2}}{2i}
\end{equation}
\begin{equation}
xz=\frac{T_{-1}-T_{+1}}{2}
\end{equation}
and 
\begin{equation}
x^2-y^2=T_{+2}+T_{-2}
\end{equation}
\item 
\begin{align}
        Q&\equiv e\braket{\alpha,j,m=j|(3z^2-r^2)|\alpha,j,m=j}\\&=e\braket{\alpha,j,m=j|(2z^2-x^2-y^2)|\alpha,j,m=j}\\&=e\sqrt{6}\braket{\alpha,j,m=j|T_0|\alpha,j,m=j}\\&=e\sqrt{6}\braket{j,2;j,0|j,2;j,j}\frac{\braket{\alpha,j||T||\alpha,j}}{\sqrt{2j+1}}
\end{align}
So, 
\begin{equation}
        \frac{\braket{\alpha,j||T||\alpha,j}}{\sqrt{2j+1}}=\frac{Q}{e\sqrt{6}\braket{j,2;j,0|j,2;j,j}}
\end{equation}
Finally, letting $F$ stand for the matrix elements we want, we have
\begin{align}
        F&=e\braket{\alpha,j,m'|(x^2-y^2)|\alpha,j,m=j}\\&=e\braket{\alpha,j,m'|T_{_2}+T_{-2}|\alpha,j,m=j}\\&=e\braket{\alpha,j,m'|T_{_2}|\alpha,j,m=j}+e\braket{\alpha,j,m'|T_{-2}|\alpha,j,m=j}\\&=e\braket{j,2;j,2|j,2;j,m'}\frac{\braket{\alpha',j||T||\alpha,j}}{\sqrt{2j+1}}+e\braket{j,2;j,-2|j,2;j,m'}\frac{\braket{\alpha',j||T||\alpha,j}}{\sqrt{2j+1}}\\&=\frac{Q}{\sqrt{6}\braket{j,2;j,0|j,2;j,j}}\left(\braket{j,2;j,2|j,2;j,m'}+\braket{j,2;j,-2|j,2;j,m'}\right)
\end{align}

\end{enumerate}

%%%%%%%%%%%%%%%%%%%%%%%%%%%%%%%%%%%%%%%%%%%%%%%%%%%%%%%%%%%%%%%%%%%%%%%%%%%%%

\item[] \textbf{S: 5.15}

The perturbation in this problem is $V = -\vec{\mu}\cdot \vec{E}$. Since $\vec{\mu}$ is proportional to the Pauli matrices, it transforms in the spin-$1/2$ representation, while $E$ transforms in the spin-1 representation. Beyond that, I'm lost and will have to ask for more direction in class.
    
%%%%%%%%%%%%%%%%%%%%%%%%%%%%%%%%%%%%%%%%%%%%%%%%%%%%%%%%%%%%%%%%%%%%%%%%%%%%%

    \item[] \textbf{S: 5.20}
    
    Let $f(x)=\braket{x|\tilde{\alpha}}=e^{-\beta|x|}$. 
    
    First, the normalization:
    \begin{align}
            \int_{-\infty}^\infty f(x)^2dx&=2\int_0^\infty e^{-2\beta x}dx\\&=2\left(-\frac{1}{2\beta}\right)\left(0-1\right)\\&=\frac{1}{\beta}
    \end{align}
    Second, the expectation value of the Hamiltonian in $\ket{\tilde{\alpha}}$:
    \begin{align}
            \braket{H}_{\alpha}&=\braket{\tilde{\alpha}|\frac{p^2}{2m}+\frac{m\omega^2x^2}{2}|\tilde{\alpha}}\\&=\int_{-\infty}^\infty f^*(x)\left(-\frac{\hbar^2}{2m}\frac{d^2}{dx^2}+\frac{m\omega^2}{2}x^2\right)f(x)dx\\&=\int_0^\infty dx\text{ }e^{-\beta x}\left(-\frac{\hbar^2}{2m}\beta^2+\frac{m\omega^2 x^2}{2}\right)e^{-\beta x}\\&+\int_{-\infty}^0 dx\text{ }e^{\beta x}\left(-\frac{\hbar^2}{2m}\beta^2+\frac{m\omega^2x^2}{2}\right)e^{\beta x}\\&=2\int_0^\infty \left(-\frac{\hbar^2}{2m}\beta^2+\frac{m\omega^2x^2}{2}\right)e^{-2\beta x}dx\\&=2\left[-\frac{\hbar^2}{2m}\beta^2\left(\frac{1}{2\beta}\right)+\frac{m\omega^2}{2}\int_0^\infty e^{-2\beta x}x^2dx\right]\\&=2\left[-\frac{\hbar^2}{4m}\beta+\frac{m\omega^2}{2}\frac{1}{4\beta^3}\right]
    \end{align}
    Actually, we are missing a term. The second derivative of our trial function is positive everywhere, except at the origin, where we can think of it as being sharply negative. To be more quantitative, 
    consider the limit
    \begin{align}
            K&=\lim_{\epsilon\to 0}\int_{-\epsilon}^\epsilon \frac{d^2}{dx^2}\left(e^{-\beta x}\right)dx\\&=\lim_{\epsilon\to0}\frac{d}{dx}\left(e^{-\beta x}\right)\biggr\rvert_{-\epsilon}^{\epsilon}\\&=-2\beta
    \end{align}
    Thus, really
    \begin{equation}
            \braket{H}_{\alpha}=\frac{\hbar^2}{2m}2\beta-\frac{\hbar^2}{2m}\beta+\frac{m\omega^2}{2}\frac{1}{4\beta^3}=\frac{\hbar^2}{2m}\beta+\frac{m\omega^2}{4\beta^3}
    \end{equation}
    
    Therefore
    \begin{equation}
            A(\beta)=\frac{\braket{H}_{\alpha}}{|f|^2}=\frac{\hbar^2}{2m}\beta^2+\frac{m\omega^2}{4}\frac{1}{\beta^2}
    \end{equation}
    
    Finally, we minimize $A$ with respect to $\beta$:
    \begin{equation}
            \frac{\partial A}{\partial \beta}=-\frac{m\omega^2}{2}\frac{1}{\beta^3}+\frac{\hbar^2}{m}\beta=0\implies \beta=\left(\frac{m^2\omega^2}{2\hbar^2}\right)^{\frac{1}{4}}
    \end{equation}
    Finally, our estimate for the ground state energy of the quantum harmonic oscillator is 
    \begin{equation}
            A(\beta)=\frac{\hbar^2}{2m}\frac{m\omega}{\sqrt{2}\hbar}+\frac{m\omega^2}{4}\frac{\sqrt{2}\hbar}{m\omega}=\boxed{\frac{\hbar \omega}{\sqrt{2}}}
    \end{equation}
    We know that the correct answer is $E_0=\frac{\hbar\omega}{2}$, so the variational method gets the right answer to within a factor of $\sqrt{2}$. 
    
%%%%%%%%%%%%%%%%%%%%%%%%%%%%%%%%%%%%%%%%%%%%%%%%%%%%%%%%%%%%%%%%%%%%%%%%%%
    
    \item[] \textbf{S: 5.21}
    Consider the differential equation $\frac{d^2\psi}{dx^2}-|x|\psi=-\lambda\psi$, $\psi\rightarrow0$ for $|x|\rightarrow\infty$, as well as the trial function:
    \begin{equation}
            \psi_\alpha= \begin{cases} 
                     c(\alpha-|x|) & \text{for} |x|<\alpha\\
                     0             & \text{for} |x|>\alpha
                    \end{cases}
    \end{equation}
    
    with the parameter $\alpha$. Thus, for a given $\alpha$:
    
    \begin{align}
            \braket{\psi_\alpha|\psi_\alpha}&=\int_{-\infty}^\infty|\psi_\alpha|^2dx\\
            &=\int_{-\alpha}^\alpha c^2(\alpha-|x|)^2dx\\
            &=2c^2\int_0^\alpha(\alpha-x)^2dx\\
            &=\frac{2}{3}c^2\alpha^3
    \end{align}
    \begin{align}
            \braket{\psi_\alpha|\frac{d^2}{dx^2}|\psi_\alpha}&=\int_{-\alpha}^\alpha c(\alpha-|x|)\frac{d^2}{dx^2}c(\alpha-|x|)dx\\
            &=c^2\int_{-\alpha}^\alpha(\alpha-|x|)\frac{d^2}{dx^2}(\alpha-|x|)dx\\
            &=-c^2\int_{-\alpha}^\alpha(\alpha-|x|)\frac{d}{dx}(2H(x)-1)dx\\
            &=-2c^2\int_{-\alpha}^\alpha(\alpha-|x|)\delta(x)dx\\
            &=-2c^2\alpha
    \end{align}
    \begin{align}
            \braket{\psi_\alpha||x||\psi_\alpha}&=\int_{-\alpha}^\alpha c(\alpha-|x|)|x|c(\alpha-|x|)dx\\
            &=c^2\int_{-\alpha}^\alpha(\alpha-|x|)^2|x|dx\\
            &=2c^2\int_{0}^\alpha(\alpha-x)^2xdx\\
            &=c^2\frac{\alpha^4}{6}
    \end{align}
    Thus, if we let $H=\frac{d^2}{dx^2}-|x|$, the lowest eigenvalue $-\lambda$ of $H$ satisfies the following inequality:
    \begin{equation}
            \lambda\geq-\frac{\braket{\psi_\alpha|H|\psi_\alpha}}{\braket{\psi_\alpha|\psi_\alpha}}=-\frac{\alpha^3-12}{4\alpha^2}
    \end{equation}
    We now try to minimize $\frac{\braket{\psi_\alpha|H|\psi_\alpha}}{\braket{\psi_\alpha|\psi_\alpha}}$:
    \begin{align}
            0&=\frac{d}{d\alpha}\frac{\braket{\psi_\alpha|H|\psi_\alpha}}{\braket{\psi_\alpha|\psi_\alpha}}\biggr\rvert_{\alpha=\alpha_0}\\
            &=\frac{1}{4}+\frac{6}{\alpha_0^3}
    \end{align}
    such that $\alpha_0=-2\cdot3^{1/3}$, and such that $\lambda\simeq-\frac{\alpha_0^3-12}{4\alpha_0^2}\simeq1.082$
    
%%%%%%%%%%%%%%%%%%%%%%%%%%%%%%%%%%%%%%%%%%%%%%%%%%%%%%%%%%%%%%%%%%%%%%%%%%

    
    
\end{enumerate}

\end{document}
